\documentclass[a4paper]{article}
\usepackage{amsmath}
\usepackage{amssymb,amsthm,amsfonts}
\usepackage[T1]{fontenc}
\usepackage[latin1]{inputenc}
\usepackage{graphicx}
\usepackage{color}
\usepackage[final]{showkeys}
\usepackage[numbered,framed]{mcode}


\title{The Fitzhugh-Nagumo System: F.D. and F.E. (by LehrFEM Library) Implementations}
\author{Frederico Santos Teixeira}

\begin{document}

\maketitle

This report presents the results of the application of the Finite Element Method in a linearized system of equations, derived from FitzHugh-Nagumo Model:
\begin{gather*}
	\partial_t u = d_u^2 \Delta u + \lambda u - \sigma v + \kappa
	\\
	\tau \partial_t v = d_v^2 \Delta v + u - v
\end{gather*}
where the domain is $ \Omega = [-1, 1]^2 $ and the coefficients are $ d_u^2 = 0.00028 $, $ lambda = 1 $, $ sigma = 1 $, $ kappa = -0.05 $, $ tau = 0.1 $ and $ d_v^2 = 0.005 $.

We are considering no-flux boundary condition, i.e. $ \frac{\partial u}{\partial n} = 0 $ and $ \frac{\partial v}{\partial n} = 0  $, along with noisy initial condition around zero (but greater than zero) for both variables.

After the variational formulation, we set $ u(x,t) = \sum_{i=1}^{N} \mu (t) . b_N^i (x) $ and  $ v(x,t) = \sum_{i=1}^{N} \nu(t) . b_N^i(x) $ plus a piecewise linear basis function over a triangular mesh to achieve:
\begin{gather*}
\overrightarrow{\mu}_t = M^{-1} A \overrightarrow{\mu} + F(\overrightarrow{\mu}, \overrightarrow{\nu})
\\
\tau \overrightarrow{\nu}_t = M^{-1} A \overrightarrow{\nu} + G(\overrightarrow{\mu}, \overrightarrow{\nu})
\end{gather*}
where $ F(\overrightarrow{\mu}, \overrightarrow{\nu}) = \lambda \overrightarrow{\mu} - \sigma \overrightarrow{\nu} + \kappa $ and $ G(\overrightarrow{\mu}, \overrightarrow{\nu}) = \overrightarrow{\mu} - \overrightarrow{\nu} $. 

The matrix $ M $ and $ A $ are the Mass and Galerkin matrices, respectively, corresponding to the basis choice.

As from now we split the procedure into two different branches: 
\begin{enumerate}
\item we create functions to compute the Mass and Galerkin matrices and we show the results using MATLAB functions;
\item we solve the whole steps using LehrFEM Library: mesh generation and refinement, assembly local elements and plot solution.
\end{enumerate}

The system can be solved by means of three solvers: "ODE45" or "ODE23S" or "SemiImplicitEuler". A flag in the code leads to the user choice.

Both codes follow, along with some solutions' pictures achieved by the three different solvers. The picture on the right represents the $u$ variable, while the picture on the left represents the $v$. Both of them were taken after 1000 time steps.

Here we present the code for the first branch:
\lstinputlisting{Codes/FHN.m}
The results follow below:

$u$ and $v$ by means of "ODE45":

\includegraphics[height=60mm]{FHN/mi_ODE45.jpg}
\includegraphics[height=60mm]{FHN/nu_ODE45.jpg}

$u$ and $v$ by means of "ODE23S":

\includegraphics[height=60mm]{FHN/mi_ODE23S.jpg}
\includegraphics[height=60mm]{FHN/nu_ODE23S.jpg}

$u$ and $v$ by means of "ImplicitEuler":

\includegraphics[height=60mm]{FHN/mi_ImplicitEuler.jpg}
\includegraphics[height=60mm]{FHN/nu_ImplicitEuler.jpg}

There are two functions which generate the Mass and Galerkin Matrices:

\lstinputlisting{Codes/MassMatrix.m}
\lstinputlisting{Codes/GalerkinMatrix.m}

Now the code for the second branch:

\lstinputlisting{Codes/FHN_LehrFEM.m}

The results follow below:

$u$ and $v$ by means of "ODE45":

\includegraphics[height=60mm]{LehrFEM/mi_ODE45.jpg}
\includegraphics[height=60mm]{LehrFEM/nu_ODE45.jpg}

$u$ and $v$ by means of "ODE23S":

\includegraphics[height=60mm]{LehrFEM/mi_ODE23S.jpg}
\includegraphics[height=60mm]{LehrFEM/nu_ODE23S.jpg}

$u$ and $v$ by means of "ImplicitEuler":

\includegraphics[height=60mm]{LehrFEM/mi_ImplicitEuler.jpg}
\includegraphics[height=60mm]{LehrFEM/nu_ImplicitEuler.jpg}

The code showed below is used to solve this system:

\lstinputlisting{Codes/nlevolution.m}

Also, fixing a initial condition and giving a time step, we can evaluate the behavior of the solution after a mesh refinement.

\begin{enumerate}

\item $u$ and $v$ for $timestep = 0.1$, $time = 50$ and $289$ elements in the mesh:

\includegraphics[height=60mm]{ImplicitEuler_1/mi289.jpg}
\includegraphics[height=60mm]{ImplicitEuler_1/nu289.jpg}

\item $u$ and $v$ for $timestep = 0.1$, $time = 50$ and $1089$ elements in the mesh:

\includegraphics[height=60mm]{ImplicitEuler_1/mi1089.jpg}
\includegraphics[height=60mm]{ImplicitEuler_1/nu1089.jpg}

\item $u$ and $v$ for $timestep = 0.1$, $time = 50$ and $4225$ elements in the mesh:

\includegraphics[height=60mm]{ImplicitEuler_1/mi4225.jpg}
\includegraphics[height=60mm]{ImplicitEuler_1/nu4225.jpg}

\item $u$ and $v$ for $timestep = 0.1$, $time = 50$ and $16641$ elements in the mesh:

\includegraphics[height=60mm]{ImplicitEuler_1/mi16641.jpg}
\includegraphics[height=60mm]{ImplicitEuler_1/nu16641.jpg}

\end{enumerate}

For the next tests we set a new time step.

\begin{enumerate}

\item $u$ and $v$ for $timestep = 0.01$, $time = 50$ and $289$ elements in the mesh:

\includegraphics[height=60mm]{ImplicitEuler_01/mi289.jpg}
\includegraphics[height=60mm]{ImplicitEuler_01/nu289.jpg}

\item $u$ and $v$ for $timestep = 0.01$, $time = 50$ and $1089$ elements in the mesh:

\includegraphics[height=60mm]{ImplicitEuler_01/mi1089.jpg}
\includegraphics[height=60mm]{ImplicitEuler_01/nu1089.jpg}

\item $u$ and $v$ for $timestep = 0.01$, $time = 50$ and $4225$ elements in the mesh:

\includegraphics[height=60mm]{ImplicitEuler_01/mi4225.jpg}
\includegraphics[height=60mm]{ImplicitEuler_01/nu4225.jpg}

\item $u$ and $v$ for $timestep = 0.01$, $time = 50$ and $16641$ elements in the mesh:

\includegraphics[height=60mm]{ImplicitEuler_01/mi16641.jpg}
\includegraphics[height=60mm]{ImplicitEuler_01/nu16641.jpg}

\end{enumerate}

We fixed the initial condition given by the function:

\lstinputlisting{Codes/InitialCondition.m}

This sort of results can be achieved for each solver described above.

\end{document}