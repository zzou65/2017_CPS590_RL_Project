%%%%%%%%%%%%%%%%%%%%%%%%%%%
%% Element mass matrices %%
%%%%%%%%%%%%%%%%%%%%%%%%%%%

\section{Element Mass Matrices} \label{sect:mass} \index{element mass matrix|(}

 The main input argument for the computation of the element mass matrices are the \ttindex{Vertices} of the current element, in most cases it is the only one. On the other hand, \ttindex{QuadRule} is needed and shape functions\index{shape functions} are called within the function to compute the element map. See table \ref{tab:stima_in} for details. Further arguments are defined when they appear. No operator is to be specified in a function handle, but e.g. weight functions which appear in the {\tt MASS\_Weight}-functions. \\

 If not specified otherwise the functions are e.g. called by \\

\noindent {\tt >> Mloc = MASS\_LFE(Vertices);} \\

 The computation of the element mass matrices \ttitindex{Mloc} is quite similar the one for the element stiffness matrices \ttindex{Aloc} described in the previous section. \\

 The element mass matrices {\tt Mloc} are $l$-by-$l$ matrices where $l$ are the local degrees of freedom. Like the element stiffness matrices they are assembled\index{assembling!global matrix} to the mass matrix\index{mass matrix} \ttindex{M} by the {\tt assemMat}-functions, p. \pageref{sect:assem_mat}.


% ueberarbeiten: input = vertices, output = Mloc, computation easy..
% weight und vol functionen einbauen

\subsection{Constant Finite Elements} \index{constant finite elements!element mass matrix}

 To call \ttitindex{MASS\_P0\_1D} and \ttitindex{MASS\_P0} only the coordinates of the vertices are needed: \\

\noindent {\tt >> Mloc = MASS\_P0\_1D(Vertices);} \\
\noindent {\tt >> Mloc = MASS\_P0(Vertices);} \\

%% weight/heat???
 The weighted versions \ttitindex{MASS\_Weight\_P0\_1D} and \ttitindex{MASS\_Weight\_P0} are called by \\

\noindent {\tt >> Mloc = MASS\_Weight\_P0\_1D(Vertices,QuadRule,FHandle,FParam);} \\
\noindent {\tt >> Mloc = MASS\_Weight\_P0(Vertices,ElemInfo,QuadRule, ... \\
FHandle,FParam);}


\subsection{Linear Finite Elements} \index{linear finite elements!element mass matrix}

\subsubsection{.. in 1D}

 \ttitindex{MASS\_P1\_1D} computes the $2$-by-$2$ element mass matrix using linear finite elemnts. The weighted version \ttitindex{MASS\_Weight\_P1\_1D} uses \ttindex{shap\_P1\_1D} and is called by \\

{\tt >> Mloc = MASS\_Weight\_P1\_1D(Vertices,QuadRule,FHandle,FParam);}

\subsubsection{.. in 2D}

 Similarily, \ttitindex{MASS\_LFE} and \ttindex{MASS\_LFE2} compute the element mass matrix in 2D. Here, {\tt Mloc} is a $3$-by-$3$ and -- in the vectorial case -- a $6$-by-$6$ matrix.
 
 


 The function \ttitindex{MASS\_Weight\_LFE} computes the element mass matrix with a given weight. It makes use of the shape functions \ttindex{shap\_LFE} and is called by \\

\noindent {\tt >> Mloc = MASS\_Weight\_LFE(Vertices,ElemInfo,QuadRule, ... \\
FHandle,FParam);} \\

 where {\tt FHandle} denotes the function handle to the weight function and {\tt FParam} its variable length argument list.


\subsection{Bilinear Finite Elements} \index{bilinear finite elements!element mass matrix}

 The $4$-by-$4$ element mass matrix using bilinear Lagrangian elements is computed by \\

\noindent {\tt >> Mloc = \ttitindex{MASS\_BFE}(Vertices,ElemInfo,QuadRule);} \\

 \ttindex{shap\_BFE} and \ttindex{grad\_shap\_BFE} are used for the computation of the element mapping.


\subsection{Crouzeix-Raviart Finite Elements} \index{Crouzeix-Raviart finite elements!element mass matrix}

 \ttitindex{MASS\_CR} computes the $3$-by-$3$ element mass matrix. \ttitindex{MASS\_Vol\_DGCR} computes the element mass matrix using discontinuous Crouzeix-Raviart finite elements.


\subsection{Quadratic Finite Elements} \index{quadratic finite elements!element mass matrix}

 The $6$-by-$6$ element mass matrix is given by \ttitindex{MASS\_QFE}.


\subsection{Whitney 1-Forms} \index{Whitney 1-forms!element mass matrix}

 \ttitindex{MASS\_W1F} computes the element mass matrix with weight {\tt MU\_Handle} for edge elements. It is specified in table \ref{tab:mu_handle}.

\begin{table}[htb]
  \begin{tabular}{p{2cm}p{9cm}}
    {\tt MU\_Handle} & {\small handle to a functions expecting a matrix whose rows represent position arguments. Return value must be a vector (variable arguments {\tt MU\_Param} will be passed to this function).} \\
  \end{tabular}
  \caption{Weight {\tt MU\_Handle}}
  \label{tab:mu_handle}
\end{table}

 The function is called by \\

\noindent {\tt >> Mloc = MASS\_W1F(Vertices,ElemInfo,MU\_Handle,QuadRule,MU\_Param);}



\subsection{$hp$ Finite Elements} \index{hpFEM@$hp$FEM!element mass matrix}

\subsubsection{.. in  1D}

 The ({\tt p}$+1$)-by-({\tt p}$+1$) element mass matrix \ttitindex{MASS\_hpDG\_1D} in the 1-dimensional case is computed by \\

\noindent {\tt >> Mloc = MASS\_hpDG\_1D(Coordinates,p,QuadRule,Shap);} \\

 \ttitindex{MASS\_Vol\_hpDG\_1D} is called by \\

\noindent {\tt >> Mloc = MASS\_Vol\_hpDG\_1D(Vertices,p,QuadRule,Shap);}

\subsubsection{.. in 2D}

 More input arguments are needed for the 2-dimensional \ttitindex{MASS\_hp}, e.g. the different polynomial degrees of the shape functions as well as the orientation of the edges. These new input arguments are listed in Table \ref{tab:mass_hp}.



 Furthermore, the pre-computed shape functions {\tt Shap} (cf. \ref{ssect:shap_hp}, p. \pageref{ssect:shap_hp}) and quadrature rules {\tt QuadRule} (cf. \ref{chap:quad_rule}, p. \pageref{chap:quad_rule}ff) are required. The mass matrix is called by \\

\noindent {\tt >> Mloc = MASS\_hp(Vertices,ElemInfo,EDofs,EDir,CDofs, ... \\
QuadRule,Shap,varargin);} \\

 and e.g. used for the routine \ttindex{main\_hp} in the folder {\tt /Examples/DiffConv\_exp}. Actually the input arguments {\tt ElemInfo} and {\tt varargin} are not needed for \linebreak
 {\tt MASS\_hp} but for \ttindex{assemMat\_hp}. \\

 The mass matrix {\tt Mloc} for the $hp$FEM is of dimension ($3+\sum \mathtt{EDof}+ \mathtt{CDof}$)-by-($3+\sum \mathtt{EDof}+ \mathtt{CDof}$).


\subsection{Mixed Finite Elements} \index{mixed finite elements!element mass matrix}

 \ttitindex{MASS\_P1P0} computes the element mass matrix using linear and constant Lagrangian finite elements. It is called by \\

\noindent {\tt >> Mloc = MASS\_P1P0(Vertices);} \\

% The only input argument needed for the function is {\tt Vertices}, but since the element mass matrices {\tt MASS\_P1P0} are assembled by {\tt assemMat\_P1P0}, more arguments are required?!

 The element mass matrix with weight is computed by \ttitindex{MASS\_Weight\_P1P0}.


\index{element mass matrix|)} 