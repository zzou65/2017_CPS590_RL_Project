%%%%%%%%%%%%%%%%%%%%%%%%%%%%%%%%
%%%%% Finite volume method %%%%%
%%%%%%%%%%%%%%%%%%%%%%%%%%%%%%%%

\chapter{Finite Volume Method} \label{chapt:fvm_conv_diff}


%%%%%%%%%%%%%%%%%%%%%%%%%%%%%%%%%%%%
%% convection/diffusion equations %%
%%%%%%%%%%%%%%%%%%%%%%%%%%%%%%%%%%%%

\section{Finite Volume Code for Solving Convection/Dif\-fusion Equations} \label{sect:fvm_conv_diff}

This section details code written for the \LIBNAME library during
the summer of 2007 by Eivind Fonn, for solving convection/diffusion equations
using the Finite Volumes approach.

\subsection{Background}

The equation in question is
\[ -\nabla\cdot(k\nabla u) + \nabla\cdot(cu) + ru = f \]
on some domain $\Omega\subset\mathbb{R}^2$. Here, $k,r,f:\Omega\to\mathbb{R}$,
and $c:\Omega\to\mathbb{R}^2$. $k$ is the diffusivity and should be positive 
everywhere. $c$ is the velocity field.

In addition to the above, one can specify Dirichlet and Neumann boundary
conditions on various parts of $\partial\Omega$.

\subsection{Mesh Generation and Plotting}

The FV mesh generation builds upon FE meshes. Given a FE mesh, use the
{\tt add\_MidPoints} function to add the data required for the dual mesh: \\

\noindent{\tt >> mesh = add\_MidPoints\index{add\_MidPoints@{\tt add\_MidPoints}}(mesh,method);} \\

{\tt method} is a string specifying which dual mesh method to use. The two
options are {\tt barycentric} and {\tt orthogonal}. If {\tt method}
is not specified, the {\tt barycentric} method will be used. For the
{\tt orthogonal} method, the mesh cannot include any obtuse triangles, and
if any such triangle exists, an error will occur.

The full mesh can be plotted using the following call: \\

\noindent{\tt >> plot\_Mesh\_LFV(mesh);} \index{plot\_Mesh\_LFV@{\tt plot\_Mesh\_LFV}}


\subsection{Local computations}

As in the case of finite elements the local stiffness matrix is computed for every element. This is done in three different functions, which will be discussed below

\subsubsection{Diffusion term}

In the function \texttt{STIMA\_GenLapl\_LFV} the local stiffness matrix comming from the diffusive term $-\nabla\cdot(k\nabla u)$ is computed. The code can be found below.

\begin{lstlisting}
function aLoc = STIMA_GenLapl_LFV(vertices,midPoints,...
      center,method,bdFlags,kHandle,varargin)
%   Copyright 2007-2007 Eivind Fonn
%   SAM - Seminar for Applied Mathematics
%   ETH-Zentrum
%   CH-8092 Zurich, Switzerland

% Compute required coefficients

m = zeros(3,3);
m(1,2) = norm(midPoints(1,:)-center);
m(1,3) = norm(midPoints(3,:)-center);
m(2,3) = norm(midPoints(2,:)-center);
m = m + m';

mu = zeros(3,3);
mu(1,2) = kHandle((midPoints(1,:)+center)/2);
mu(1,3) = kHandle((midPoints(3,:)+center)/2);
mu(2,3) = kHandle((midPoints(2,:)+center)/2);
mu = mu + mu';

mum = mu.*m;

sigma = zeros(3,2);
sigma(1,:) = confw(novec(vertices([2 3],:)),vertices(1,:)-vertices(2,:));
sigma(2,:) = confw(novec(vertices([1 3],:)),vertices(2,:)-vertices(3,:));
sigma(3,:) = confw(novec(vertices([1 2],:)),vertices(3,:)-vertices(1,:));
sigma(1,:) = sigma(1,:)/trialtv(vertices);
sigma(2,:) = sigma(2,:)/trialtv([vertices(2,:);vertices([1 3],:)]);
sigma(3,:) = sigma(3,:)/trialtv([vertices(3,:);vertices(1:2,:)]);

% Compute stiffness matrix

aLoc = zeros(3,3);
for i=1:3
   for j=1:3
      % Calculate aLoc(i,j)
      for l=[1:(i-1) (i+1):3]
         aLoc(i,j) = aLoc(i,j) -...
         mum(i,l)*dot(sigma(j,:),...
           confw(novec([midPoints(midpt([i l]),:);center]),...
           vertices(l,:)-vertices(i,:)));
      end
   end
end
	
% Helping functions

function s = midpt(idx)
   idx = sort(idx);
   if idx==[1 2]
      s = 1;
   elseif idx==[2 3]
      s = 2;
   else
      s = 3;
   end
end

function v = confw(v,t)
   if dot(t,v)<0
      v = -v;
   end
end

function out = novec(v)
   out = v(2,:)-v(1,:);
   if norm(out) ~= 0
      out = [-out(2) out(1)]/norm(out);
   end
end

function out = trialtv(v)
	out = norm(v(2,:)-v(1,:) + ...
	      dot(v(3,:)-v(2,:),v(1,:)-v(2,:))/...
	      (norm(v(3,:)-v(2,:))^2)*(v(3,:)-v(2,:)));
end

end
\end{lstlisting}

In the lines 10-14 the lengths of the pieces of the dual mesh on the current element are computed. In the lines 16-20 the approximation of the function $k$ on the boundary of the dual element is computed. The lines 24-30 compute the gradient of the basis functions on the current element. Finally the stiffness matrix is computed in the lines 34-45. Therefore the approximation of the integrals over the pieces of boundary of the dual mesh are added using the computed gradients of the basis functions.

In the lines 47-77 helping functions are implemented. They are used to compute the gradient and to find the correct number of the midpoint lying on an edge.

For the discretization of the convective term the function \texttt{STIMA\_GenGrad\_LFV} is used. The code of which can be found below.

\begin{lstlisting}
function aLoc = STIMA_GenGrad_LFV(vertices,midPoints,center,method,...
		bdFlags,cHandle,kHandle,rHandle,conDom,varargin)
%   Copyright 2007-2007 Eivind Fonn
%   SAM - Seminar for Applied Mathematics
%   ETH-Zentrum
%   CH-8092 Zurich, Switzerland

% Compute required coefficients

m = zeros(3,3);
m(1,2) = norm(midPoints(1,:)-center);
m(1,3) = norm(midPoints(3,:)-center);
m(2,3) = norm(midPoints(2,:)-center);
m = m + m';

qr = gauleg(0, 1, 4, 1e-6);
c = zeros(3,2);
for i=1:length(qr.w)
    c(1,:) = c(1,:) + qr.w(i)*cHandle(midPoints(1,:)+...
       	qr.x(i)*(center-midPoints(1,:)));
    c(2,:) = c(2,:) + qr.w(i)*cHandle(midPoints(2,:)+...
       	qr.x(i)*(center-midPoints(2,:)));
    c(3,:) = c(3,:) + qr.w(i)*cHandle(midPoints(3,:)+...
       	qr.x(i)*(center-midPoints(3,:)));
end

gamma = zeros(3,3);
gamma(1,2) = dot(confw(novec([midPoints(1,:); center]),...
	vertices(2,:)-vertices(1,:)),c(1,:));
gamma(2,3) = dot(confw(novec([midPoints(2,:); center]),...
	vertices(3,:)-vertices(2,:)),c(2,:));
gamma(1,3) = dot(confw(novec([midPoints(3,:); center]),...
	vertices(3,:)-vertices(1,:)),c(3,:));
gamma = gamma - gamma';

if conDom
    mu = ones(3,3);
    mu(1,2) = kHandle((midPoints(1,:)+center)/2);
    mu(2,3) = kHandle((midPoints(2,:)+center)/2);
    mu(1,3) = kHandle((midPoints(3,:)+center)/2);
    mu = mu + mu';

    d = zeros(3,3);
    d(1,2) = norm(vertices(1,:)-vertices(2,:));
    d(2,3) = norm(vertices(2,:)-vertices(3,:));
    d(1,3) = norm(vertices(3,:)-vertices(1,:));
    d = d + d';

    z = gamma.*d./mu;
    r = zeros(3,3);
    for i=1:3, for j=1:3
        r(i,j) = rHandle(z(i,j));
    end; end;
else
    r = ones(3,3)/2;
end

a = m.*gamma.*r;
b = m.*gamma;

% Compute stiffness matrix

aLoc = b'-a';

for i=1:3
	aLoc(i,i) = sum(a(i,:));
end

aLoc = aLoc';

% Helping functions

function v = confw(v,t)
	if dot(v,t)<0
		v = -v;
	end
end

function out = novec(v)
	out = v(2,:)-v(1,:);
        if norm(out) ~= 0
            out = [-out(2) out(1)]/norm(out);
        end
end
	
end
\end{lstlisting}

In the lines 10-14 the lengths of the pieces of the dual mesh on the current element are computed. 



\subsection{Assembly}

Assembly of the stiffness matrices and load vector is done much the same way
as in FE, i.e. assemble the stiffness matrices term-by-term, sum them up to get
the full matrix. Assemble the load vector in the normal way, then incorporate
Neumann boundary data, followed by Dirichlet boundary data.

For the lazy, there is a wrapper function for all of the above: \\

\noindent{\tt >> [A,U,L,fd] = assemMat\_CD\_LFV(mesh,k,c,r,d,n,f,cd,out);} \index{assemMat\_CD\_LFV@{\tt assemMat\_CD\_LFV}} \\

This function assembles all the matrices and vectors required at once, without
any hassle. Here follows a list of the input arguments:

\begin{itemize}
\item {\tt mesh} The mesh struct (see last section).
\item {\tt k} Function handle for the $k$-function.
\item {\tt c} Function handle for the $c$-function.
\item {\tt r} Function handle for the $r$-function.
\item {\tt d} Function handle for the Dirichlet boundary data.
\item {\tt n} Function handle for the Neumann boundary data.
\item {\tt f} Function handle for the $f$-function.
\item {\tt cd} If equal to 1, specifies that the problem is convection-dominated,
and that appropriate upwinding techniques should be applied. If 0, specifies
that the problem is diffusion-dominated. If left out, the program itself will
determine whether the problem is convection- or diffusion-dominated.
\item {\tt out} If equal to 1, the program will write output to the console.
This is useful for large meshes, to keep track of the program.
\end{itemize}

If any of the function handles are left out or set equal to the empty matrix, 
it is assumed they are zero and will play no part.

The output are as follows:

\begin{itemize}
\item {\tt A} The stiffness matrix.
\item {\tt U} The solution vector, containing Dirichlet boundary data.
\item {\tt L} The load vector.
\item {\tt fd} A vector with the indices of the free nodes.
\end{itemize}

From the above, the solution can be obtained by: \\

\noindent{\tt >> U(fd) = A(fd,fd)$\backslash$L(fd);} \\

The solution vector {\tt U} contains the approximate values of the solution $u$
at the nodes given in the mesh.

\subsection{Error Analysis}

Given a mesh {\tt mesh}, a FV solution vector {\tt U} and a function handle 
{\tt u} to the exact solution, the error can be measured in three different
norms ($\|\cdot\|_1$, $\|\cdot\|_2$ and $\|\cdot\|_\infty$) using the 
following calls: \\

\noindent{\tt >> L1err = \ttindex{L1Err\_LFV}(mesh,U,qr,u);} \\
\noindent{\tt >> L2err = \ttindex{L2Err\_LFV}(mesh,U,qr,u);} \\
\noindent{\tt >> Linferr = \ttindex{LInfErr\_LFV}(mesh,U,u);} \\

Here, {\tt qr} is any quadrature rule for the reference element, i.e.
{\tt qr = \ttindex{P7O6}()}. 

\subsection{File-by-File Description}

Here follows a list of the files which were written as part of this project,
and a description of each.

\begin{itemize}
\item {\tt Assembly/\ttitindex{assemDir\_LFV}.m} \\
    {\tt [U,fd] = assemDir\_LFV(mesh,bdFlags,fHandle)} incorporates the
    Dirichlet boundary conditions given by the function {\tt fHandle} in
    the Finite Volume solution vector {\tt U}. {\tt fd} is a vector
    giving the indices of the vertices with no Dirichlet boundary data
    (i.e. the free vertices).
\item {\tt Assembly/\ttitindex{assemLoad\_LFV}.m} \\
    {\tt L = assemLoad\_LFV(mesh,fHandle)} assembles the load vector {\tt L}
    for the data given by the function {\tt fHandle}.
\item {\tt Assembly/\ttitindex{assemMat\_CD\_LFV}.m} \\
    Wrapper function for assembly of convection/diffusion problems. See
    above for description.
\item {\tt Assembly/\ttitindex{assemMat\_LFV}.m}  \\
    {\tt A = assemMat\_LFV(mesh,fHandle,varargin)} assembles the global
    stiffness matrix {\tt A} from the local element contributions given
    by the function {\tt fHandle} and returns it in a sparse format.
    {\tt fHandle} is passed the extra input arguments given by {\tt varargin}.
\item {\tt Assembly/\ttitindex{assemNeu\_LFV}.m} \\
    {\tt L = assemNeu\_LFV(mesh,bdFlags,L,qr,fHandle)} incorporates Neumann
    boundary conditions in the load vector {\tt L} as given by the function
    {\tt fHandle}. {\tt qr} is any 1D quadrature rule. Note: In the general
    diffusion problem $-\nabla\cdot(k\nabla u)=f$, with Neumann boundary data
    $\frac{\partial u}{\partial n}=g$, the function $h$ you need to pass to
    the Neumann assembly function is $h(x)=g(x)k(x)$.
\item {\tt Element/\ttitindex{STIMA\_GenGrad\_LFV}.m} \\
    {\tt A = STIMA\_GenGrad\_LFV(v,mp,cp,m,bd,cH,kH,rH,cd)} calculates the
    local element contribution to the stiffness matrix from the term
    $\nabla\cdot(cu)$. Here, {\tt v} is a 3-by-2 matrix giving the vertices
    of the element, {\tt mp} and {\tt cp} give the midpoints on each of the
    edges as well as the centerpoint (that is, the dual mesh geometry).
    {\tt m} is the method used to generate the dual mesh (not used at the
    current time), {\tt bd} are boundary flags for the edges (not used
    at the current time). {\tt cH} is the $c$-function, {\tt kH} is the
    $k$-function from the diffusion term and {\tt rH} is a proper upwinding
    function. {\tt cd} is 1 if the problem is convection-dominated and 0
    otherwise. {\tt kH} and {\tt rH} are only used if {\tt cd=1}. If not,
    they need not be specified.
\item {\tt Element/\ttitindex{STIMA\_GenLapl\_LFV}.m} \\
    {\tt A = STIMA\_GenLapl\_LFV(v,mp,cp,m,bd,kH)} calculates the local
    element contribution to the stiffness matrix from the term
    $-\nabla\cdot(k\nabla u)$. The arguments are the same as above.
\item {\tt Element/\ttitindex{STIMA\_ZerOrd\_LFV}.m} \\
    {\tt A = STIMA\_ZerOrd\_LFV(v,mp,cp,m,bd,rH)} calculates the local
    element contribution to the stiffness matrix from the term
    $ru$. The arguments are the same as above, except that {\tt rH} is
    the function handle for the $r$-function.
\item {\tt Errors/\ttitindex{L1Err\_LFV}.m} \\
    {\tt e = L1Err\_LFV(mesh,U,qr,fHandle)} calculates the discretization
    error in the norm $\|\cdot\|_1$, between the finite volume approximation
    given by {\tt U} (the solution vector), and the exact solution given by
    the function {\tt fHandle}. {\tt qr} is any appropriate quadrature rule
    on the reference element.
\item {\tt Errors/\ttitindex{L2Err\_LFV}.m} \label{L2Err_LFV} \\
    {\tt e = L2Err\_LFV(mesh,U,qr,fHandle)} calculates the discretization
    error in the norm $\|\cdot\|_2$. All the arguments are as above.
\item {\tt Errors/\ttitindex{LInfErr\_LFV}.m} \label{LInfErr_LFV} \\
    {\tt e = LInfErr\_LFV(mesh,U,fHandle)} calculates the discretization
    error in the norm $\|\cdot\|_\infty$. All the arguments are as above.
\item {\tt MeshGen/\ttitindex{add\_MidPoints}.m} \\
    {\tt mesh = add\_MidPoints(mesh,method)} incorporates the dual mesh
    data required for the finite volume method into the mesh {\tt mesh}.
    {\tt method} may be either {\tt 'barycentric'} or {\tt 'orthogonal'}.
    See earlier description.
\item {\tt Plots/\ttitindex{plot\_Mesh\_LFV}.m} \\
    {\tt plot\_Mesh\_LFV(mesh)} plots the base mesh and the dual mesh.
\end{itemize}


\subsection{Driver routines}

The driver routines for solving convection diffusion equations can be found in the folder \texttt{/Examples/FVOL}. There are $3$ different example problems implemented, which we shall discuss in more detail. The files \texttt{exp\_*} are functions, which solve the problems for the given input parameters. The functions \texttt{exp\_run\_*} contain executeable code and call the corresponding \texttt{exp\_*} function to solve the problem for different settings. Below the code of \texttt{exp\_run\_1} is included.

\subsubsection{Example 1}

\begin{lstlisting}
exp1_epsrange = [1e-4 5e-4 1e-3 5e-3 1e-2 5e-2 1e-1 5e-1];
refrange = 1:6;
exp1_meshes = [];
exp1_solutions = [];
exp1_errors_L1 = [];
exp1_errors_L2 = [];
exp1_errors_Linf = [];
exp1_hrange = [];

for r=1:size(refrange,2)
    for e=1:size(exp1_epsrange,2)
        disp(['Running test ' num2str((e-1)+...
              size(exp1_epsrange,2)*(r-1)+1) ' of ' ...
              num2str(size(exp1_epsrange,2)*size(refrange,2))]);
        [errs, mw, msh, u] =...
              exp_1(exp1_epsrange(e), refrange(r), 0, 0);
        exp1_errors_L1(e,r) = errs(1);
        exp1_errors_L2(e,r) = errs(2);
        exp1_errors_Linf(e,r) = errs(3);
        exp1_solutions(e,r).sol = u;
    end

    exp1_meshes(r).Coordinates = msh.Coordinates;
    exp1_meshes(r).Elements = msh.Elements;
    exp1_meshes(r).ElemFlag = msh.ElemFlag;
    exp1_meshes(r).Edges = msh.Edges;
    exp1_meshes(r).Vert2Edge = msh.Vert2Edge;
    exp1_meshes(r).Max_Nodes = msh.Max_Nodes;
    exp1_meshes(r).BdFlags = msh.BdFlags;
    exp1_meshes(r).CenterPoints = msh.CenterPoints;
    exp1_meshes(r).Type = msh.Type;
    exp1_meshes(r).MidPoints = msh.MidPoints;
    exp1_hrange(r) = mw;
end

    

save 'exp1_lfv.mat' exp1_*
\end{lstlisting}

There are two loops that start in the lines 10 and 11 control the variable $\epsilon$ and the number of mesh refinements respectively. A call to the function \texttt{exp\_1} in line 15 computes the finite volume solution of the equation
\begin{equation}
 -\epsilon\Delta u + u_x = 1,
\end{equation}
on the triangle $\Omega = \{(x,y)\;|\;0\leq x\leq1,\;-x\leq y\leq x\}$. In every step different values for $\epsilon$ and different mesh widths (refinement steps) are handed to \texttt{eps\_1}. The other two input parameters are flags, which are here not set to suppress output and plotting. The code of the function \texttt{eps\_1} will be explained in more detail below.

In the last line all the results including errors, solution and mesh information are saved to the file \texttt{exp1\_lfv.mat}.

Now we shall discuss the solver routine for this example, the code can be found below.

\begin{lstlisting}
function [errs, mw, msh, U] = exp_1(eps, nRef, plotting, output)
if nargin < 2 || isempty(nRef)
    nRef = 0;
end
if nargin < 3 || isempty(plotting)
    plotting = 0;
end
if nargin < 4 || isempty(output)
    output = 0;
end
    
out('Generating mesh.');
msh = load_Mesh('mesh_exp_1_coords.dat', 'mesh_exp_1_elements.dat');
msh.ElemFlag = ones(size(msh.Elements,1),1);
msh = add_Edges(msh);
loc = get_BdEdges(msh);
msh.BdFlags = zeros(size(msh.Edges,1),1);
msh.BdFlags(loc) = -ones(size(loc));
for i=1:nRef
    msh = refine_REG(msh);
end
msh = add_MidPoints(msh, 'barycentric');

u = @(x,varargin)(x(1)-(exp(-(1-x(1))/eps)-...
	exp(-1/eps))/(1-exp(-1/eps)));
k = @(x,varargin)eps;
c = @(x,varargin)[1 0];
r = [];
d = @(x,varargin)(u(x,varargin));
n = [];
f = @(x,varargin)1;
conDom = 1;

[A, U, L, fd] =...
    assemMat_CD_LFV(msh, k, c, r, d, n, f, conDom, output);
  
out('Solving system.');
U(fd) = A(fd,fd)\L(fd);

if plotting 
    out('Plotting.');
    plot_LFE(U, msh);
end

out('Calculating errors.');
errs = [L1Err_LFV(msh, U, P7O6(), u);...
     L2Err_LFV(msh, U, P7O6(), u); LInfErr_LFV(msh, U, u)];
out(['Discretization errors: ' num2str(errs(1)) ',...
    ' num2str(errs(2)) ', ' num2str(errs(3)) '.']);

out('Calculating meshwidth.');
mw = get_MeshWidth(msh);


% Helping functions
function out(text, level)
    if output
        if nargin < 2 || isempty(level),
            level = 1;
        end
        for i=1:level
            text = ['- ' text];
        end
        disp([text]);
    end
end
end

\end{lstlisting}

\begin{itemize}
 \item lines 2-10: the output and plotting flags are set, if they are not part of the input already
 \item lines 12-22: generate and refine mesh, in line 22 the midpoints for the finite volume method are computed
 \item lines 23-32: define all the input variables for the finite volume method, i.e. the functions for defining the differential equation and for defining the Dirichlet and Neumann data; the variable \texttt{conDom} is a flag specifying if the problem is convergence dominated
 \item lines 40-45: plot the solution if the \texttt{plotting} flag is set
 \item lines 47-54: compute the discretization errors and print them in case the flag \texttt{output} is set
 \item lines 57-68: the function \texttt{out} controls the output in the command window
\end{itemize}

Furthermore there the function \texttt{main\_1} solves uses the \texttt{eps\_1} function to solve one example problem. The input parameters can be given to the function in the beginning. The solution and the underlying mesh will be plotted.

\subsubsection{Example 2}

The structure of Example 2 is the same as in the first one. The equation that is solved is given by
\begin{equation}
 -\epsilon\Delta u + u_x + u_y= 0,
\end{equation}
with dirichlet boundary condition on the square $[0,1]^2$. The exact solution we are looking for is in this case given by
\begin{equation}
 u(x,y)=
 \begin{cases}
  1 \qquad 	&\text{if $x>y$}\\
  0.5		&\text{if $x=y$}\\
  0		&\text{if $x<y$}.
 \end{cases}
\end{equation}
On the boundary Dirichlet conditions are set.

\subsubsection{Example 3}

For the third example there is no wrapper function \texttt{eps\_3\_run} but only the function \texttt{eps\_3}, which solves the problem
\begin{equation}
 -\Delta u + \nabla\cdot(u\nabla\Psi) = 0,
\end{equation}
where $\Psi=\Psi(r)=\tfrac{1}{1+e^{a(r-1)}}$. The radius is given by $r=\sqrt{x^2+y^2}$ and the computational domain is given by $[0,1]^2$. The last two input parameters of the function are used to control plotting and output. The input variable \texttt{a} is needed in the definition of the function $\Psi$. The variable \texttt{nRef} determines the number of refinement steps.


For this example there is, as for the experiments before the function \texttt{main\_3}, which solves the problem for the parameters specified in the beginning of the file.
