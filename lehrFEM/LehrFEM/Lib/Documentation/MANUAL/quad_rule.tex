%%%%%%%%%%%%%%%%%%%%%%%%%
%% Quadrature formulas %%
%%%%%%%%%%%%%%%%%%%%%%%%%

\chapter{Numerical Integration} \label{chap:quad_rule} \index{quadrature rules|(} \index{integration!numerical|see{quadrature rules}}

 Numerical integration is needed for the computation of the load vector in \ref{sect:assem_load}, the incorporation of the Neumann boundary conditions in \ref{sect:assem_neu} and in some cases -- e.g. for the $hp$FEM -- also for the computation of the element stiffness matrices (if the differential operator doesn't permit analytic integration) and the incorporation of the Dirichlet boundary conditions. \\

 The quadrature rules in \LIBNAME are stored in the folder {\tt /Lib/QuadRules}. There are 1D and 2D quadrature rules implemented and listed below. For some basic formulas of numerical integration see for example \cite{AS64}, chapter 25.


%%%%% Data structure %%%%%%

\section{Data Structure of Quadrature Rules} \index{quadrature rules!data structure}

 A $Q$-point quadrature rule \ttitindex{QuadRule} computes a quadrature rule on a standard reference element. The \MATLAB struct contains the fields weights {\tt w} and abscissae {\tt x}. They are specified in table \ref{tab:quad_rule}.

\begin{table}[htb]
  \begin{tabular}{p{0.5cm}p{10.5cm}}
	{\tt w} & {\small $Q$-by-$1$ matrix specifying the weights of the quadrature rule} \\
    	{\tt x} & {\small $Q$-by-$1$ (1D) or $Q$-by-$2$ (2D) matrix specifying the abscissae of the quadrature rule}
  \end{tabular}
  \caption{Quadrature rule structure \index{quadrature rules!data structure}}
  \label{tab:quad_rule}
\end{table}

 In the following sections {\tt QuadRule\_1D} and {\tt QuadRule\_2D} are used instead of {\tt QuadRule} to highlight their dimension. \\

 The barycentric coordinates {\tt xbar} of the quadrature points {\tt x} may be recovered by \\

\noindent {\tt >> xbar = [Quadrule.x, 1-sum(QuadRule.x,2)];} \\


% overview: http://mathworld.wolfram.com/topics/NumericalIntegration.html (1D) and http://www.iopb.res.in/~somen/abramowitz_and_stegun

%%%%% 1D quadrature rules %%%%%

\section{1D Quadrature Rules} \label{sect:quad_rule_1d} \index{quadrature rules!1-dimensional|textit}

 The 1D quadrature rules are used for the incorporation of the boundary conditions and generally for 1D problems. The 1D Gauss-Legendre\index{quadrature rules!1-dimensional!Gauss-Legendre|textit}\index{Gauss-Legendre quadrature rule} quadrature rule \ttitindex{gauleg} and the 1D Legendre-Gauss-Lobatto\index{quadrature rules!1-dimensional!Gauss-Legendre-Lobatto|textit}\index{Gauss-Legendre-Lobatto quadrature rule} quadrature rule \ttitindex{gaulob} are implemented in LehrFEM. \\

 They are called by \\

\noindent {\tt >> QuadRule\_1D = gauleg(a,b,n,tol);} \\
\noindent {\tt >> QuadRule\_1D = gaulob(a,b,n,tol);} \\

 and compute the respective {\tt n}-point quadrature rules on the interval {\tt [a,b]}. The prescribed tolerance {\tt tol} determines the accuracy of the computed integration points. If no tolerance is prescribed the machine precision {\tt eps} is used. \\

 All orders of the quadrature rules {\tt gauleg} are of order $2${\tt n}$-1$, the ones of {\tt gaulob} are of order $2${\tt n}$-3$. The abscissas for quadrature order {\tt n} are given by the roots of the Legendre polynomials $P_{\mathtt{n}}(x)$. In the Lobatte quadrature the two endpoints of the interval are included as well. \\

 The 1D quadrature rules may be transformed to rules on squares resp. triangles by {\tt TProd} resp. {\tt TProd} and {\tt Duffy}. See \ref{ssect:quad_trans} below.


%%%%% other functions %%%%%

%\subsection{\textcolor{pink}{Transformations and other modifications?}}

%{\tt TProd} and {\tt Duffy}


%%%%% 2D quadrature rules %%%%%

\section{2D Quadrature Rules} \label{sect:quad_rule_2d} \index{quadrature rules!2-dimensional|textit}

 In two dimension two reference elements can be distinguished -- the unit square $[0,1]^2$ and the triangle with the vertices $(0,0)$, $(1,0)$, and $(0,1)$. There are also two different ways to build quadrature rules -- from 1D quadrature rules or from scratch. Both approaches are used in the \LIBNAME and are examined in the following.

\subsection{Transformed 1D Quadrature Rules} \label{ssect:quad_trans} \index{quadrature rules!2-dimensional!transformed|textit}

 The quadrature formula for the unit square are Gaussian quadrature rules which are the tensorized version of 1-dimensional formulas. To this end the tensor product\index{tensor product} \ttitindex{TProd} is applied by \\

\noindent {\tt >> QuadRule\_2D\_square = TProd(QuadRule\_1D);} \\

 where {\tt QuadRule\_1D} is a 1-dimensional quadrature rule of \ref{sect:quad_rule_1d}, i.e. {\tt gauleg} or {\tt gaulob}. This type of integration is used for finite elements defined on sqares such as $[0,1]^2$, e.g. bilinear finite elements. \\

 % e.g. used in main_Heat_BFE etc.

 Furthermore, by the use of the Duffy transformation\index{Duffy transformation} \ttitindex{Duffy} of the integration points and the weights one obtains a 2-dimensional quadrature rule for triangular elements: \\

\noindent {\tt >> QuadRule\_2D\_triangle = Duffy(TProd(QuadRule\_1D));}

The $i$-th integration point then has the coordinates $x_{i,1}$ and $x_{i,2}(1-x_{i,1})$, where $x_{i,1}$ and $x_{i,2}$ are the two coordinates of the non transformed point. Furtheremore the corresponding weights are transformed according to $w_i(1-x_{i,1})$.

% e.g. used in main_hp for hpFEM

\subsection{2D Gaussian Quadrature Rules} \index{quadrature rules!2-dimensional!Gaussian|textit} \index{Gaussian quadrature rule, 2D} \label{ssect:quad_po}

 Several Gaussian quadrature rules on the above mentioned reference triangle are implemented. Their file names are of the form {\tt PnOo}, which stands for '{\tt n}-point quadrature rule of order {\tt O}', e.g. {\tt P4O3}.

These quadrature rules do not need any input since the number of points and the integration domain, i.e. the unit triangle are specified.\\

 The quadrature rules are called by  e.g.\\

\noindent {\tt >> QuadRule\_2D\_triangle = P4O3();} \\

 So far the following quadrature rules are implemented in LehrFEM: \ttitindex{P1O2}, \ttitindex{P3O2}, \ttitindex{P3O3}, \ttitindex{P4O3}, \ttitindex{P6O4}, \ttitindex{P7O4}, \ttitindex{P7O6}, \ttitindex{P10O4} and \ttitindex{P10O5}.


\subsection{2D Newton-Cotes Quadrature Rules} \index{quadrature rules!2-dimensional!Newton-Cotes|textit} \index{Newton-Cotes quadrature rule}

 The Newton-Coates quadrature rule for the reference triangle is implemented in {\tt /Lib/QuadRules/private}. Here \ttitindex{ncc\_triangle\_rule} is called by \\

\noindent {\tt >> nc = ncc\_triangle\_rule(o,n);} \\

 where {\tt o} is the order and {\tt n} the number of points. Because the output of this function doesn't have the right format, the program \ttitindex{NCC} tranforms it. After all \\

\noindent {\tt >> QuadRule\_2D\_triangle = NCC(o);} \\

 provides the right data structure as specified in table \ref{tab:quad_rule} for {\tt n}$=10$.

\index{quadrature rules|)}