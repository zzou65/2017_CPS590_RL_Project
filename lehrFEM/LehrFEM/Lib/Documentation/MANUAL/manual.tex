\documentclass[a4paper,10pt]{book}

% Caption of figures shall be centered and not as wide as the pagewidth
\def\capwidth{0.85\textwidth}
\def\capsize{\small}

% Imported packages
\usepackage{amssymb}
\usepackage{amsmath}
\usepackage{nicefrac}
\usepackage[dvips,dvipdf]{graphicx}
\usepackage{epsfig}
% \usepackage{bibcontents} % includes bibliography in the table of contents (bibcontents.sty needed)
\usepackage{makeidx} % to make an index
\usepackage{fancyhdr} % for a fancy header
\usepackage{listings}
\usepackage[usenames,dvipsnames]{color}
% This is the color used for MATLAB comments below
\definecolor{MyDarkGreen}{rgb}{0.0,0.4,0.0}

%%%%%%%%%%%%%%%%%%%%%%%%%%%%%%%%%%%%%%%%%%%%%%%%%%%%%%%%%%%%%%%%%%%%%%
% Enable this for viewing in xdvi 
\usepackage[hypertex]{hyperref}
\usepackage{showkeys}

%%%%%%%%%%%%%%%%%%%%%%%%%%%%%%%%%%%%%%%%%%%%%%%%%%%%%%%%%%%%%%%%%%%%%%
% Enable this for PDF output
%\usepackage[[dvips,bookmarks=true,linkcolor=green]{hyperref}

% \hypersetup{colorlinks=false}
% run: 1. tex->dvi, 2. dvi->ps, 3. ps->pdf 





\makeindex

% Color text
\usepackage{color}
\definecolor{pink}{rgb}{1,0,1}

% short cuts to put words with typewriter style in index (without typing it 3 times)
\newcommand{\ttindex}[1]{{\tt #1}\index{#1@{\tt #1}}}
\newcommand{\ttitindex}[1]{{\tt #1}\index{#1@{\tt #1}|textit}}


% Name definitions
\newcommand{\LIBNAME}{LehrFEM }
\newcommand{\MATLAB}{MATLAB }



% Page headings
% \pagestyle{myheadings}
%\markright{\small\scshape \LIBNAME Manual}
%\pagenumbering{arabic}
% \pagestyle{fancyplain}

% Page style
\pagestyle{fancy}
\renewcommand{\chaptermark}[1]%
   {\markboth{\textsc{\thechapter.\ #1}}{}}
\renewcommand{\sectionmark}[1]%
   {\markright{\thesection.\ #1}}
\renewcommand{\headrulewidth}{0.5pt}
\renewcommand{\footrulewidth}{0pt}
\newcommand{\helv}{%
   \fontsize{9}{11}\selectfont}
\fancyhf{}
\fancyhead[LE,RO]{\helv \thepage}
\fancyhead[LO]{\helv \rightmark}
\fancyhead[RE]{\helv \leftmark}

% Define titlepage
\title{\LIBNAME - A 2D Finite Element Toolbox}
\author{Annegret Burtscher, Eivind Fonn, Patrick Meury}
\date{\today}



% 'Introduction' and 'Mesh Generation' by Patrick Meury, Jun 2005 - original files in /Lib/Documentation
% 'Finite volume code for solving convection/diffusion equations' by Eivind Fonn, Summer 2007
% 'Overview', 'Basis functions', 'Numerical Integration', 'Local computations', 'Assembling', 'Boundary conditions', 'Plotting the solution' and 'Discretization errors' by Annegret Burtscher, Jul-Sep 2008


\begin{document}
\lstloadlanguages{Matlab}%
\lstset{language=Matlab,                        % Use MATLAB
        frame=single,                           % Single frame around code
        basicstyle=\small\ttfamily,             % Use small true type font
        keywordstyle=[1]\color{blue}\bf,        % MATLAB functions bold and blue
        keywordstyle=[2]\color{purple},         % MATLAB function arguments purple
        keywordstyle=[3]\color{blue}\underbar,  % User functions underlined and blue
        identifierstyle=,                       % Nothing special about identifiers
                                                % Comments small dark green courier
        commentstyle=\usefont{T1}{pcr}{m}{sl}\color{MyDarkGreen}\small,
        stringstyle=\color{Purple},             % Strings are purple
        showstringspaces=false,                 % Don't put marks in string spaces
        tabsize=5,                              % 5 spaces per tab
        %
        %%% Put standard MATLAB functions not included in the default
        %%% language here
        morekeywords={xlim,ylim,var,alpha,factorial,poissrnd,normpdf,normcdf},
        %
        %%% Put MATLAB function parameters here
        morekeywords=[2]{on, off, interp},
        %
        %%% Put user defined functions here
        morekeywords=[3]{FindESS, homework_example},
        %
        morecomment=[l][\color{Blue}]{...},     % Line continuation (...) like blue comment
        numbers=left,                           % Line numbers on left
        firstnumber=1,                          % Line numbers start with line 1
        numberstyle=\tiny\color{Blue},          % Line numbers are blue
        stepnumber=5                            % Line numbers go in steps of 5
        }


% Make titlepage
\maketitle

% Make table of contents
\tableofcontents
 

% Include chapter 'Intro'
\input intro
\addcontentsline{toc}{chapter}{Introduction}

% Incluce chapter 'Overview'
\input overview
\addcontentsline{toc}{chapter}{Overview}

% Include chapter 'Mesh generation'
\input mesh_gen

% Include chapter 'Basis functions'
\input shap_fct

% Include chapter 'Numerical integration'
\input quad_rule

% Include chapter 'Local computations'
\input stima
\input mass
\input load

% Include chapter 'Assembling'
\input assem_mat
\input assem_load

% Include chapter 'Boundary conditions'
\input assem_dir
\input assem_neu

% Include chapter 'Computation of solution'


% What about interpolation, e.g. the functions LFE_DOF_interp, QFE_DOF_interp etc.


% Include chapter 'Plot'
\input plot

% Include chapter 'Discretization errors'
\input err
\input h1_err
\input h1s_err
\input l1_err
\input l2_err
\input linf_err



% Include chapter 'Examples'
\input ex

% Include chapter 'Finite volume method'
\input fvm_conv_diff



% Make the bibliography
\newpage
\addcontentsline{toc}{chapter}{Bibliography}
\bibliography{bibfile}
\bibliographystyle{alpha}
%\bibliographystyle{unsrt}

% Make Index
\newpage
\addcontentsline{toc}{chapter}{Index}
\printindex


\end{document}
